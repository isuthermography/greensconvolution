\documentclass[letterpaper]{article}
\usepackage{amsmath}

\newcommand{\cedilla}{\text{\c{c}}}

\begin{document}

The 1D thermal Green's function for an impulse source on the boundary of a half-space is 
\begin{equation}
\frac{2}{\rho c \left[4\pi (k/(\rho c))t\right]^{1/2}}\exp\left(-\frac{z^{2}}{4 (k/(\rho c)) t}\right)   \ \ \ \ \ \ (t > 0)
\end{equation}
( This is twice the full-space green's function: half of the energy flows into the material -- the other half flows in the other direction where we don't really care about it).

The 3D thermal Green's function for an impulse source on the boundary of a half-space is likewise 
\begin{equation}
\frac{2}{\rho c \left[4\pi (k/(\rho c))t\right]^{3/2}}\exp\left(-\frac{|r|^{2}}{4 (k/(\rho c)) t}\right)  \ \ \ \ \ \ (t > 0)
\end{equation}

The effective heat source of a buried delamination superimposes with the 1D greens function of the flash to give zero heat flux in the $z$ direction at the delamination.

Therefore the heat source is $k \frac{\partial T}{\partial z}$ as a 1D $g(z,t)$, for a half-space greens's function 

This derivative of the 1D Green's function is:
\begin{equation}
k\frac{\partial T}{\partial z}=\frac{2k}{\rho c \left[4\pi (k/(\rho c))t\right]^{1/2}}\exp\left(-\frac{z^{2}}{4 (k/(\rho c)) t}\right)\left(-\frac{z}{2(k/(\rho c))t}\right)  \ \ \ \ \ \ (t > 0)
\end{equation}


The temporal convolution of this source with the the 3D Green's function is
coefficients and substituting $\alpha$ for $k/(\rho c)$ is
\begin{equation}
-\frac{4kz}{2\rho^{2} c^{2}\alpha\pi^{2}} \int_{-\infty}^{\infty}\frac{1}{\tau}\frac{1}{\left[4\alpha\tau\right]^{1/2}}\exp\left(-\frac{z^{2}}{4\alpha\tau}\right)\frac{1}{\left[4\alpha(t-\tau)\right]^{3/2}}\exp\left(-\frac{|r|^{2}}{4\alpha(t-\tau)}\right)
\end{equation}
where the integrand is defined to be zero where $\tau < 0$ or $(t-\tau) < 0$.
Pull in the $\tau$ in the denominator and a $4\alpha$ changing the 1/2 to 3/2:
\begin{equation}
-\frac{8\alpha z}{\rho c\pi^{2}} \int_{-\infty}^{\infty}\frac{1}{\left[4\alpha\tau\right]^{3/2}}\exp\left(-\frac{z^{2}}{4\alpha\tau}\right)\frac{1}{\left[4\alpha(t-\tau)\right]^{3/2}}\exp\left(-\frac{|r|^{2}}{4\alpha(t-\tau)}\right)
\end{equation}

The constraint on the integrand can be converted to bounds on the integration:
\begin{equation}
-\frac{8\alpha z}{\rho c\pi^{2}} \int_{0}^{t}\frac{1}{\left[4\alpha\tau\right]^{3/2}}\exp\left(-\frac{z^{2}}{4\alpha\tau}\right)\frac{1}{\left[4\alpha(t-\tau)\right]^{3/2}}\exp\left(-\frac{|r|^{2}}{4\alpha(t-\tau)}\right) d\tau
\end{equation}
This integral can be nondimensionalized by substituting $u=4\alpha\tau/z^{2}$, $v=4\alpha t/z^{2}$,  and $\cedilla=|r|/|z|$. Therefore
$(v-u)=(4\alpha/z^{2})(t-\tau)$, $(v-u)/\cedilla^{2}=(4\alpha/|r|^{2})(t-\tau)$,
and $du=(4\alpha/z^{2})d\tau$
\begin{equation}
-\frac{8\alpha z}{\rho c\pi^{2}}\frac{z^{2}}{4\alpha}\int_{0}^{v}\frac{1}{\left[uz^{2}\right]^{3/2}}\exp\left(-\frac{1}{u}\right) \frac{1}{\left[(v-u)(|r|^{2}/\cedilla^{2})\right]^{3/2}}\exp\left(-\frac{\cedilla^{2}}{v-u}\right) du
\end{equation}
or 
\begin{equation}
-\frac{8\alpha z}{\rho c\pi^{2}}\frac{z^{2}}{4\alpha}\frac{1}{|z|^{3}\left[(|r|^{2}/\cedilla^{2})\right]^{3/2}}\int_{0}^{v}\frac{1}{\left[u\right]^{3/2}\left[v-u\right]^{3/2}}\exp\left(-\frac{1}{u}-\frac{\cedilla^{2}}{v-u}\right) du
\end{equation}
or, canceling leading variables
\begin{equation}
-\frac{2}{\rho c\pi^{2}}  \frac{1}{\left[(|r|^{2}/\cedilla^{2})\right]^{3/2}}\int_{0}^{v}\frac{1}{\left[u\right]^{3/2}\left[v-u\right]^{3/2}}\exp\left(-\frac{1}{u}-\frac{\cedilla^{2}}{v-u}\right) du
\end{equation}
... and since $|r|/\cedilla = z$,
\begin{equation}
-\frac{2}{\rho c\pi^{2}z^{3}} \int_{0}^{v}\frac{1}{\left[u\right]^{3/2}\left[v-u\right]^{3/2}}\exp\left(-\frac{1}{u}-\frac{\cedilla^{2}}{v-u}\right) du
\end{equation}

%NOTE: P defined two distinct $c$ variables: heat capacity (in leading coefficeint) vs. nondimensionalization parameter. DO NOT CONFUSE... SHOULD FIX!

Now consider extreme case where $\cedilla^{2} >>  v$ and $\cedilla^{2} >> 1$:
Recall $\cedilla^{2}/v = |r|^{2}/(4\alpha t)$
Rewrite as 
\begin{equation}
-\frac{2}{\rho c\pi^{2}z^{3}} \int_{0}^{v}\frac{1}{\left[u\right]^{3/2}\left[v-u\right]^{3/2}}\exp\left(-\frac{1}{u}-\frac{1}{v-u}-\frac{(\cedilla^{2}-1)}{v-u}\right) du
\end{equation}
... Must be no greater than
\begin{equation}
-\frac{2}{\rho c\pi^{2}z^{3}} \int_{0}^{v}\frac{1}{\left[u\right]^{3/2}\left[v-u\right]^{3/2}}\exp\left(-\frac{1}{u}-\frac{1}{v-u}\right) du \exp\left(-\frac{(\cedilla^{2}-1)}{v}\right)
\end{equation}
The integral is the same as our previous integral,
but at $\cedilla=1$. This never exceeds 0.185
so a simple upper bound on the value of the expression is
\begin{equation}
-\frac{2}{\rho c\pi^{2}z^{3}}0.185*\exp\left(-\frac{(\cedilla^{2}-1)}{v}\right)
\end{equation}

\section{Extension to anisotropic media aligned with coordinate axes}
$k$ now has three components (actually a diagonal tensor),
$k_{x}$,$k_{y}$, and $k_{z}$.

The 1D thermal Green's function for a boundary impulse:
\begin{equation}
\frac{2}{\rho c \left[4\pi (k_{z}/(\rho c))t\right]^{1/2}}\exp\left(-\frac{z^{2}}{4 (k_{z}/(\rho c)) t}\right)   \ \ \ \ \ \ (t > 0)
\end{equation}
The 3D thermal Green's function for a boundary impulse is the product
of three 1D solutions:
\begin{equation}
\frac{2}{\rho c k_{x}^{1/2}k_{y}^{1/2}k_{z}^{1/2}\left[4\pi (1/(\rho c))t\right]^{3/2}}\exp\left(-\frac{|x|^{2}}{4 (k_{x}/(\rho c)) t} -\frac{|y|^{2}}{4 (k_{y}/(\rho c)) t} -\frac{|z|^{2}}{4 (k_{z}/(\rho c)) t} \right)  \ \ \ \ \ \ (t > 0)
\end{equation}

Derivative of the 1D Green's function is:
\begin{equation}
k_{z}\frac{\partial T}{\partial z}=\frac{2k_{z}}{\rho c \left[4\pi (k_{z}/(\rho c))t\right]^{1/2}}\exp\left(-\frac{z^{2}}{4 (k_{z}/(\rho c)) t}\right)\left(-\frac{z}{2(k_{z}/(\rho c))t}\right)  \ \ \ \ \ \ (t > 0)
\end{equation}

The temporal convolution of this source with the 3D Green's function is
 substituting $\alpha_{z}$ for $k_{z}/(\rho c)$ is
\begin{equation}
  -\frac{4k_{z}z}{2\rho^{2} c^{2}\alpha_{z}\pi^{2}} \int_{-\infty}^{\infty}\frac{1}{\tau}\frac{1}{\left[4\alpha_{z}\tau\right]^{1/2}}\exp\left(-\frac{z^{2}}{4\alpha_{z}\tau}\right)\frac{1}{k_{x}^{1/2}k_{y}^{1/2}k_{z}^{1/2}\left[4(1/(\rho c))(t-\tau)\right]^{3/2}}\exp\left(-\frac{|x_{m}|^{2}}{4 (k_{x}/(\rho c)) (t-\tau)} -\frac{|y_{m}|^{2}}{4 (k_{y}/(\rho c)) (t-\tau)} -\frac{|z_{m}|^{2}}{4 (k_{z}/(\rho c)) (t-\tau)}\right)
  \label{eq:3danisok}
\end{equation}
at measurement coordinates $(x_{m},y_{m},z_{m})$ relative to the reflector position
Define $\alpha_{x} = k_{x}/\rho c$, $\alpha_{y} = k_{y}/\rho c$, and
$\alpha_{xyz}=\sqrt[3]{k_{x}k_{y}k_{z}/(\rho c)^{3}}$:
\begin{equation}
-\frac{4k_{z}z}{2\rho^{2} c^{2}\alpha_{z}\pi^{2}} \int_{-\infty}^{\infty}\frac{1}{\tau}\frac{1}{\left[4\alpha_{z}\tau\right]^{1/2}}\exp\left(-\frac{z^{2}}{4\alpha_{z}\tau}\right)\frac{1}{\left[4\alpha_{xyz}(t-\tau)\right]^{3/2}}\exp\left(-\frac{|x_{m}|^{2}}{4 \alpha_{x} (t-\tau)} -\frac{|y_{m}|^{2}}{4 \alpha_{y} (t-\tau)} -\frac{|z_{m}|^{2}}{4 \alpha_{z} (t-\tau)}\right)
  \label{eq:3danisoalpha}
\end{equation}


Pull in the tau in denominator and a $4 \alpha_{z}$: 
\begin{equation}
-\frac{8\alpha_{z}z}{\rho c\pi^{2}} \int_{-\infty}^{\infty}\frac{1}{\left[4\alpha_{z}\tau\right]^{3/2}}\exp\left(-\frac{z^{2}}{4\alpha_{z}\tau}\right)\frac{1}{\left[4(\alpha_{xyz}(t-\tau)\right]^{3/2}}\exp\left(-\frac{|x_{m}|^{2}}{4 \alpha_{x} (t-\tau)} -\frac{|y_{m}|^{2}}{4 \alpha_{y} (t-\tau)} -\frac{|z_{m}|^{2}}{4 \alpha_{z} (t-\tau)}\right)
\end{equation}
Set integral bounds
\begin{equation}
-\frac{8\alpha_{z}z}{\rho c\pi^{2}} \int_{0}^{t}\frac{1}{\left[4\alpha_{z}\tau\right]^{3/2}}\exp\left(-\frac{z^{2}}{4\alpha_{z}\tau}\right)\frac{1}{\left[4(\alpha_{xyz}(t-\tau)\right]^{3/2}}\exp\left(-\frac{|x_{m}|^{2}}{4 \alpha_{x} (t-\tau)} -\frac{|y_{m}|^{2}}{4 \alpha_{y} (t-\tau)} -\frac{|z_{m}|^{2}}{4 \alpha_{z} (t-\tau)}\right) d\tau
\end{equation}
%Assume transverse isotropy ($k_{x}$=$k_{y}$=$k_{xy}$ therefore $\alpha_{x}=\alpha_{y}=\alpha_{xy}$).
Define $\cedilla=(\sqrt{x_{m}^{2}(\alpha_{z}/\alpha_{x})+y_{m}^{2}(\alpha_{z}/\alpha_{y})+z_{m}^{2}}/|z|)$
\\
Observe:
\begin{equation}
\left(-\frac{|x_{m}|^{2}}{4 \alpha_{x} (t-\tau)} -\frac{|y_{m}|^{2}}{4 \alpha_{y} (t-\tau)} -\frac{|z_{m}|^{2}}{4 \alpha_{z} (t-\tau)}\right)
\end{equation}
\begin{equation}
=\left(-\frac{|x_{m}|^{2}(\alpha_{z}/\alpha_{x})}{4 \alpha_{z} (t-\tau)} -\frac{|y_{m}|^{2}(\alpha_{z}/\alpha_{y})}{4 \alpha_{z} (t-\tau)} -\frac{|z_{m}|^{2}(\alpha_{z}/\alpha_{z})}{4 \alpha_{z} (t-\tau)}\right)
\end{equation}
\begin{equation}
=\left(-\frac{|x_{m}|^{2}(\alpha_{z}/\alpha_{x}) + |y_{m}|^{2}(\alpha_{z}/\alpha_{y}) + |z_{m}|^{2}(\alpha_{z}/\alpha_{z})}{4 \alpha_{z} (t-\tau)}\right)
\end{equation}
\begin{equation}
=\left(\frac{-\cedilla^{2} |z|^{2}}{4 \alpha_{z} (t-\tau)}\right)
\end{equation}

Nondimensionalize everythig by substituting $u=4\alpha_{z}\tau/z^{2}$, $v=4\alpha_{z} t/z^{2}$,  and . Therefore
$(v-u)=(4\alpha_{z}/z^{2})(t-\tau)$, $(\alpha_{xyz}/\alpha_{z})(v-u)=(4\alpha_{xyz}/z^{2})(t-\tau)$ and $du=(4\alpha_{z}/z^{2})d\tau$.
The exponent
\begin{equation}
\left(\frac{-\cedilla^{2} |z|^{2}}{4 \alpha_{z} (t-\tau)}\right)
\end{equation}
becomes
\begin{equation}
\left(\frac{-\cedilla^{2}}{v-u}\right)
\end{equation}
%and
%\begin{equation}
%\frac{\alpha_{z}}{\alpha_{xyz}}u = 4\alpha_{z}\tau/z^{2}
%\end{equation}
The original expression can now be reduced to:
\begin{equation}
-\frac{2 z\alpha_{z}^{3/2}}{\rho c\pi^{2}\alpha_{xyz}^{3/2}z^{4}} \int_{0}^{v}\frac{1}{\left[u\right]^{3/2}}\exp\left(-\frac{1}{u}\right)\frac{1}{\left[(v-u)\right]^{3/2}}\exp\left(\frac{-\cedilla^{2}}{v-u}\right) du
\end{equation}
Coalesce:
\begin{equation}
-\frac{2 \alpha_{z}^{3/2}}{\rho c\pi^{2}\alpha_{xyz}^{3/2}z^{3}} \int_{0}^{v}\frac{1}{(v-u)^{3/2}\left[u\right]^{3/2}}\exp\left(-\frac{1}{u} - \frac{\cedilla^{2}}{v-u}\right) du
\end{equation}
Alternate representation: Integral from 0...1.\\
let $s=u/v$. Now $du = v ds$ and $s=1$ when $u=v$
\begin{equation}
-\frac{2 \alpha_{z}^{3/2}}{\rho c\pi^{2}\alpha_{xyz}^{3/2}z^{3}} \int_{0}^{1}\frac{1}{(v-vx)^{3/2}\left[vs\right]^{3/2}}\exp\left(-\frac{1}{vs} - \frac{\cedilla^{2}}{v-vs}\right) v ds
\end{equation}
\begin{equation}
-\frac{2 \alpha_{z}^{3/2}}{\rho c\pi^{2}\alpha_{xyz}^{3/2}z^{3}v^{2}} \int_{0}^{1}\frac{1}{(1-s)^{3/2}\left[s\right]^{3/2}}\exp\left(-\frac{1-s + s\cedilla^{2}}{vs(1-s)}\right) ds
\end{equation}
let $a=\cedilla^{2}-1$
\begin{equation}
-\frac{2 \alpha_{z}^{3/2}}{\rho c\pi^{2}\alpha_{xyz}^{3/2}z^{3}v^{2}} \int_{0}^{1}\frac{1}{(1-s)^{3/2}\left[s\right]^{3/2}}\exp\left(-\frac{1 + as}{vs(1-s)}\right) ds
\end{equation}

...

%ExtraVolumeFactor = 0.25*Pred_kx[:,:,:]*np.sqrt(np.pi*alphaz*Pred_t[:,:,:])
Considering the alternative scenario (curved) where there
is a leading factor in the approximate 3D Green's function of the form
\[
\frac{1}{1+\frac{1}{4}\kappa \sqrt{\pi\alpha_{z}t}},
\]
where $\kappa$ is the curvature. The $\frac{1}{4}\kappa \sqrt{\pi\alpha_{z}t}$ term in the denominator is bounded to be not smaller than -0.6 nor greater than +1.0 based on empirical observation of the quality of the approximation. 

In the concave case there are additional factors multiplied by the $|x_{m}|^{2}$ term in the exponent, of
\begin{equation}
(1 + z\kappa) \ \ \ \mbox{and}\ \ \ \left(1+\frac{1}{12}\theta^{2}\right)^{2\left(1-\frac{z}{|x||\theta/2 + \theta^{3}/24|}\right)}
\end{equation}
where $\theta \equiv \kappa x$ is bounded to not exceed $\pi/2$ in magnitude
and the second factor (the ``deceleration'') is bounded to not be lower than 1.0.

In the convex case the $|x_{m}|^{2}$ term in the exponent is replaced
by 
\begin{equation}
  \left( -\frac{((1/|\kappa|) - z_{m})^{2}(\alpha_{z}/\alpha_{x})\left(1+\frac{|\kappa| z}{1-|\kappa| z}\right)\left(4\mbox{ins}^{2}(|\kappa|x_{m}/2)\right)}{4\alpha_{z}(t-\tau)}   \right)
\end{equation}
where
$\mbox{ins}^{2}(x)$ (``inner sin squared'') is defined as
\begin{equation}
\mbox{ins}^{2}(x) \equiv \begin{cases} \sin^{2}(x) & |x| < \pi/4.0 \\ 0.25 + (|x| - \pi/4 + 0.5)^{2} & \mbox{otherwise} \end{cases} 
\end{equation}

Continuing as in Eq.~\ref{eq:3danisoalpha},
\begin{equation}
-\frac{4k_{z}z}{2\rho^{2} c^{2}\alpha_{z}\pi^{2}} \int_{-\infty}^{\infty}\frac{1}{\tau}\frac{1}{\left[4\alpha_{z}\tau\right]^{1/2}}\exp\left(-\frac{z^{2}}{4\alpha_{z}\tau}\right)\frac{1}{1+\frac{1}{4}\kappa \sqrt{\pi\alpha_{z}(t-\tau)}}\frac{1}{\left[4\alpha_{xyz}(t-\tau)\right]^{3/2}}\exp\left( ... \right)
\end{equation}


Pull in the tau in denominator and a $4 \alpha_{z}$: 
\begin{equation}
-\frac{8\alpha_{z}z}{\rho c\pi^{2}} \int_{-\infty}^{\infty}\frac{1}{\left[4\alpha_{z}\tau\right]^{3/2}}\exp\left(-\frac{z^{2}}{4\alpha_{z}\tau}\right)\frac{1}{1+\frac{1}{4}\kappa \sqrt{\pi\alpha_{z}(t-\tau)}}\frac{1}{\left[4(\alpha_{xyz}(t-\tau)\right]^{3/2}}\exp\left( ... \right)
\end{equation}
Set integral bounds
\begin{equation}
-\frac{8\alpha_{z}z}{\rho c\pi^{2}} \int_{0}^{t}\frac{1}{\left[4\alpha_{z}\tau\right]^{3/2}}\exp\left(-\frac{z^{2}}{4\alpha_{z}\tau}\right)\frac{1}{1+\frac{1}{4}\kappa \sqrt{\pi\alpha_{z}(t-\tau)}}\frac{1}{\left[4(\alpha_{xyz}(t-\tau)\right]^{3/2}}\exp\left( ... \right) d\tau
\end{equation}
Performing similar subtitutions,
where
\begin{equation}
  \cedilla=\begin{cases}
  \sqrt{x_{m}^{2}(1 + z\kappa)\left(1+\frac{1}{12}\theta^{2}\right)^{2\left(1-\frac{z}{|x||\theta/2 + \theta^{3}/24|}\right)}(\alpha_{z}/\alpha_{x})+y_{m}^{2}(\alpha_{z}/\alpha_{y})+z_{m}^{2}}/|z| & \mbox{(concave)} \\
  \sqrt{(\left[1/|\kappa|\right] - z_{m})^{2}\left(1+\frac{|\kappa| z}{1-|\kappa| z}\right)\left(4\mbox{ins}^{2}(|\kappa|x_{m}/2)\right)(\frac{\alpha_{z}}{\alpha_{x}})+y_{m}^{2}(\frac{\alpha_{z}}{\alpha_{y}})+z_{m}^{2}}/|z| & \mbox{(convex)} \\
  \end{cases}
  \end{equation}
(where as before $|\theta|$ is bounded to not exceed $\pi/2$ and the $\left(1+\frac{1}{12}\theta^{2}\right)^{2\left(1-\frac{z}{|x||\theta/2 + \theta^{3}/24|}\right)}$ factor in the concave case is bounded to be not less than 1.0)
and in  addition
$w \equiv \kappa z \sqrt{\pi}/8$:
\begin{equation}
-\frac{2 z\alpha_{z}^{3/2}}{\rho c\pi^{2}\alpha_{xyz}^{3/2}z^{4}} \int_{0}^{v}\frac{1}{\left[u\right]^{3/2}}\exp\left(-\frac{1}{u}\right)\frac{1}{1+w \sqrt{(v-u)}}\frac{1}{\left[(v-u)\right]^{3/2}}\exp\left(\frac{-\cedilla^{2}}{v-u}\right) du
\end{equation}

Alternate representation: Integral from 0...1.\\
let $s=u/v$. Now $du = v ds$ and $s=1$ when $u=v$; let $a=\cedilla^{2}-1$
\begin{equation}
-\frac{2 \alpha_{z}^{3/2}}{\rho c\pi^{2}\alpha_{xyz}^{3/2}z^{3}v^{2}} \int_{0}^{1}\frac{1}{(1-s)^{3/2}\left[s\right]^{3/2}}\frac{1}{1+w \sqrt{v(1-s)}}\exp\left(-\frac{1 + as}{vs(1-s)}\right) ds
\end{equation}
Recall that the $w\sqrt{v(1-s)}$ term in the denominator is bounded to be not smaller than -0.6 nor greater than +1.0

A not-so-great approxmation is to take the divisor of $1+w \sqrt{v(1-s)}$
and factor it out in front of the integral. For large $v$ the
integrand (see plot\_integrand.py, esp. last figure) has a very sharp peak
at s=0 which will have a sifting property replacing $s$ by $0$, enabling us to
factor it out as
\begin{equation}
-\frac{2 \alpha_{z}^{3/2}}{\rho c\pi^{2}\alpha_{xyz}^{3/2}z^{3}v^{2}} \frac{1}{1+w \sqrt{v}} \int_{0}^{1}\frac{1}{(1-s)^{3/2}\left[s\right]^{3/2}}\exp\left(-\frac{1 + as}{vs(1-s)}\right) ds
\end{equation}

At c=1 (meaning directly above the source) and smaller $v$ the integrand peak is in the middle at s=0.5 which would give us

\begin{equation}
-\frac{2 \alpha_{z}^{3/2}}{\rho c\pi^{2}\alpha_{xyz}^{3/2}z^{3}v^{2}} \frac{1}{1+w \sqrt{v/2}} \int_{0}^{1}\frac{1}{(1-s)^{3/2}\left[s\right]^{3/2}}\exp\left(-\frac{1 + as}{vs(1-s)}\right) ds
\end{equation}

Given that the latter case is going to be more dominant in the greensinversion
application we will approximate between $1$ and $1/\sqrt{2}$ as $0.8$: 

\begin{equation}
-\frac{2 \alpha_{z}^{3/2}}{\rho c\pi^{2}\alpha_{xyz}^{3/2}z^{3}v^{2}} \frac{1}{1+0.8 w \sqrt{v}} \int_{0}^{1}\frac{1}{(1-s)^{3/2}\left[s\right]^{3/2}}\exp\left(-\frac{1 + as}{vs(1-s)}\right) ds
\end{equation}
where $w\sqrt{v}$ is bounded to be not smaller than -0.6 nor greater than +1.0.

The result of this approximation keeps the integral a function of only
two variables ($v$ and $c$ -- or equivalently $v$ and $a$) and brings
the third parameter $w$ out in front as a leading coefficient. The simpler integral is easier to calculate
by look-up table interpolation. 

\end{document}
